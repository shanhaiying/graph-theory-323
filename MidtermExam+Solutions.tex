\documentclass[10pt,reqno]{amsart}
\usepackage{amsmath}
\usepackage{amsthm}
\usepackage{amssymb}
\usepackage{graphicx}
\usepackage{mathrsfs}
\usepackage{color}
\usepackage{stmaryrd}
\usepackage{hyperref}
\usepackage{eucal}
\usepackage{fullpage}
\usepackage{outlines}

\renewcommand{\labelitemi}{$\star$}

\begin{document}

\noindent \textit{Math 323: Graph Theory}

\noindent \textit{Instructor: Sreekar M.~Shastry}

\noindent \textit{Solutions to the Midterm Examination}

\noindent \textit{2013-March-02-Sat, 0900-1030 in room C304 of HR4}

\medskip

\begin{itemize}
    \item There are 6 problems. Each problem is worth 5 points. The
        maximum score is 30 points.
    \item This is an open book exam.
    \item Clearly state the results from the book that you use, referring to
      the page number.
\end{itemize}

\medskip

\begin{outline}[enumerate]
\1 Prove that every set of six people contains at least three mutual
acquaintances or three mutual strangers.

\medskip
\noindent \emph{Solution.} Let $G$ be a graph with six vertices, one for each
person, and an edge for each pair of people who are acquainted. Let $x$ be a
vertex. Since $x$ has at most 5 neighbors, $x$ has at least 3 neighbors or at
least 3 nonneighbors. We may assume without loss of generality that $x$ has at
least 3 neighbors (if not, replace $G$ with $\overline{G}$ (= the complement of
$G$)). If any pair of these people are acquainted, then taking them together
with $x$ gives us a $K_3$ as an induced subgraph in $G$, or in other words, we
have three mutual acquaintances. On the other hand, if no pair of the neighbors
of $x$ is acquainted, then the neighbors of $x$ contain a set of three mutual
strangers.

In purely graph theoretical terms, this problem says that any graph on six
vertices has $K_3$ or $\overline{K}_3$ as an induced subgraph.

\medskip
\1 Let $G$ be a graph with $|G| \ge 2.$ Show that $G$ has two vertices of equal
degree.

\medskip
\noindent \emph{Solution.} The degree of a vertex in a graph with $n$ vertices
is an element of \(\{0,1,\dots,n-1\}\). These are $n$ distinct values. Suppose
that no two degrees are the same so that $0,1,\dots,n-1$ is the degree sequence
of $G$. But a vertex of degree zero is isolated (not connected to any other
vertex), and a vertex of degree $n-1$ is connected to every other vertex. These
two conditions are contradictory.

\medskip

\medskip
\1 A saturated hydrocarbon $C_k H_l$ is a molecule formed from $k$ carbon atoms
and $l$ hydrogen atoms by adding bonds between atoms such that each carbon atom
is in four bonds, each hydrogen atom is in one bond, and no sequence of bonds
forms a cycle. Show that $l = 2k+2.$

\medskip
\noindent \emph{Solution.} Let $G$ be the graph made from the molecule, with
vertices for the atoms and edges for the bonds. Thus $|G| = k+l$. The condition
that there is no cycle means that $G$ is a tree, so $|E(G)| = k+l-1$. Now we
use the fact that the sum of the degrees is equal to twice the number of edges
to obtain \(4k+l = 2(k+l-1)\) and therefore $l=2k+2.$
\medskip

\medskip
\1 Let $G$ be a weighted graph and let $T$ be a minimum spanning tree. Let $T'$
be any other spanning tree, not necessarily of minimum weight. Show that $T'$
can be transformed into $T$ by steps that exchange one edge of $T'$ for one
edge of $T$ such that the edge set is always a spanning tree and the total
weight never goes up.

\medskip
\noindent \emph{Solution.}
The idea is simple: pick an edge $e'$ in $T'$ which is not in $T$ and then
carefully choose an edge $e$ in $T$ so that the weight of $T'-e'+e$ doesn't go
up. After finitely many steps, this procedure will transform $T'$ into $T.$ Now
let us give the details of the proof.

It suffices to find one step in the transformation from $T'$ to $T$. For if
that has been done, then the entire sequence of steps exists by induction on
the number of edges in which the two trees differ.

Fix $e' \in E(T') - E(T)$. Then $T'-e' = U \sqcup V$ has two connected
components. The path in $T$ between the endpoints of $e'$ must have an edge $e$
from $U$ to $V$. Since $e$ is an edge of the path in $T$ between the endpoints
of $e'$, the edge $e$ belongs to the unique cycle in $T+e'$. Thus $T+e'-e$ is
also a spanning tree. The proof will be complete if we show \(w(T'-e'+e) \le
w(T').\)

Since $T$ has minimum weight, $w(T) \le w(T-e'+e)$, so we have $w(e) \le
w(e')$. Therefore $w(T'-e'+e) \le w(T')$, as required.

\medskip

\medskip
\1 Show that if $G'$ is obtained from a connected graph $G$ by adding edges
joining pairs of vertices whose distance in $G$ is 2, then $G'$ is 2-connected.

\medskip
\noindent \emph{Solution.} Let us first note that $G'$ is connected since it is
made by adding edges to the connected graph $G.$

To show that $G'$ is 2-connected we must show that any vertex cut must have at
least two elements. Suppose not. Then $G'$ has a cut vertex $v$ and $G'-v$ is
disconnected.

Now, since $G'$ is connected, $G'-v$ is disconnected if and only if some of the
neighbors of $v$ in $G'$ are in one component of $G'-v$ and some are in another
component of $G'-v$.

By definition of $G'$, the neighbors of $v$ in $G$ are adjacent in $G'$ and
therefore they are in the same component of $G'-v$. Hence the neighbors of $v$
in $G'$ are in the same component of $G'-v$ and hence $G'-v$ is connected, a
contradiction.

\medskip

\medskip
\1 Show that if a connected graph $G$ has blocks $B_1,B_2,\dots,B_k$, then
\[|G| = \left( \sum_{i=1}^k |B_i| \right)- k + 1.\]

\medskip
\noindent \emph{Solution.} Let us give two proofs.

First proof. It follows from corollary 5.9 on page 113 of the book that two
distinct blocks have at most one vertex in common and that vertex is a cut
vertex.

Another way to look at what corollary 5.9 is as follows. If we form a new graph
$G^*$ with one vertex for each block of $G$ and an edge between vertices $v_1$
and $v_2$ if and only if $B_1\cap B_2\neq\varnothing$, where $v_i$ corresponds
to $B_i$, then the graph $G^*$ is a tree on $k$ vertices. This is because
corollary 5.9(c) means that each edge of $G^*$ is a bridge, and therefore $G^*$
is a tree by Theorem 4.1 on page 86.

We use the fact that a tree on $k$ vertices has $k-1$ edges to see that there
are $k-1$ cut vertices.

Therefore if we form the sum \(\sum_{i=1}^k |B_i| \) we have counted each cut
vertex twice. So we subtract the number of cut vertices to obtain \( (
\sum_{i=1}^k |B_i| )- k + 1\), as required.

Second proof. We proceed by induction on $k$. If $k=1$ then $|G| = |B_1|$.
Assume the result is true for all graphs with $k-1$ blocks. We must prove it
for a graph $G$ with $k$ blocks.

The fact that $k > 1$ means that $\kappa(G) = 1$ because deleting any cut
vertex will result in a disconnected graph. Thus $G$ is not 2-connected, so
there is a block $B$ which contains only one of the cut vertices (this block
corresponds to an end vertex of the tree $G^*$ constructed in the first proof,
and such end vertices exist in trees by 4.3 on page 89 of the book). Write $v$
for this cut vertex. We may choose notation so that $B = B_k$ is the highest
numbered block.

Put $G' := G - (V(B_k) - \{v\})$. The graph $G'$ is connected and has blocks
$B_1,\dots,B_{k-1}.$ By induction, we have $|G'| =
(\sum_{i=1}^{k-1}|B_i|)-(k-1)+1.$ Since we deleted $|B_k|-1$ vertices from $G$
to obtain $G'$, we have \( |G| = ( \sum_{i=1}^k |B_i| )- k + 1\), as required.

(Secretly, this proof reproduces the inductive proof that a tree on $n$
vertices has $n-1$ edges.)

\end{outline}

\end{document}
