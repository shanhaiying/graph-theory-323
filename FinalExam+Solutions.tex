\documentclass[10pt,reqno]{amsart}
\usepackage{amsmath}
\usepackage{amsthm}
\usepackage{amssymb}
\usepackage{graphicx}
\usepackage{mathrsfs}
\usepackage{color}
\usepackage{stmaryrd}
\usepackage{hyperref}
\usepackage{eucal}
\usepackage{fullpage}
\usepackage{outlines}

\renewcommand{\labelitemi}{$\star$}

%%%%%%%%%%%%%%%%%%%%%%%%%%%%%%%%%%%%%%%%%%%%%%%%%%%%%%%%%%%%%%%%%%%%%%%%%%%%%%%%

\begin{document}

\noindent \textit{Math 323: Graph Theory}

\noindent \textit{Instructor: Sreekar M.~Shastry}

\noindent \textit{Solutions to the Final Examination}

\noindent \textit{2013-April-19-Fri, 0900-1200 in room C304 of HR4}

\medskip

\begin{itemize}
    \item There are 10 problems. Each problem is worth 5 points. The
        maximum score is 50 points.
    \item This is an open book exam.
    \item Clearly state the results from the book that you use, referring to
      the page number.
\end{itemize}

\medskip

\begin{outline}[enumerate]
%%%%%%%%%%%%%%%%%%%%%%%%%%%%%%%%%%%%%%%%%%%%%%%%%%%%%%%%%%%%%%%%%%%%%%%%%%%%%%%%
\1 Show that $K_{n,n}$ has $n!(n-1)!/2$ Hamiltonian cycles.

\medskip
\noindent \emph{Solution.}
Since we are working with the \emph{complete} bipartite graph, in order to give
a Hamiltonian cycle, it suffices to give an ordering in which the vertices of
each partite set will be visited. In other words, we must count the number of
sequences of the form \[u_1 v_1 u_2 v_2 \cdots u_n v_n\] where the $u_i$ are in
one partition and the $v_i$ are in the other; this is because the completeness
guarantees us that once we have chosen the first $k$ vertices of the cycle, we
can choose any of the remaining unvisited vertices in the other bipartition,
and by the time we reach $v_n$, there will be an unused edge from $v_n$ back to
$u_1$ to complete the cycle.

Now, without loss of generality, we fix our start vertex $u_1.$ There are $n!$
ways to order the $v_i$ and $(n-1)!$ ways to order $u_2, u_3, \dots, u_n$. But
we have counted each Hamiltonian cycle twice since a cycle can be traversed in
either direction. Thus the total number of Hamiltonian cycles is $n!(n-1)!/2.$

\medskip
%%%%%%%%%%%%%%%%%%%%%%%%%%%%%%%%%%%%%%%%%%%%%%%%%%%%%%%%%%%%%%%%%%%%%%%%%%%%%%%%
\1 Show that in a league with two divisions of 13 teams each, no schedule has
each team playing exactly nine games against teams in its own division and four
games against teams in the other division.

\medskip
\noindent \emph{Solution.} Suppose that there existed such a schedule of games.
Then we form the graph $G$ with a vertex for each team and an edge for each
game. Then the subgraph $H$ induced by the 13 teams in one division is
9-regular. But we know that $H$ cannot exist, since in an $r$ regular graph on
$n$ vertices, one of $r$ and $n$ must be even.

\medskip
%%%%%%%%%%%%%%%%%%%%%%%%%%%%%%%%%%%%%%%%%%%%%%%%%%%%%%%%%%%%%%%%%%%%%%%%%%%%%%%%
\1 Show that if $n \ge 2$ and $d_1, d_2, \dots, d_n$ are positive integers with
$d_1 \ge d_2 \ge \cdots \ge d_n,$ then there exists a tree with these as its
vertex degrees if and only if $d_n = 1$ and $\sum_{i=1}^n d_i = 2(n-1).$

\medskip
\noindent \emph{Solution.} (I misstated the problem. The condition should have
included $d_n = d_{n-1} = 1,$ not just $d_n=1.$ The problem is still true as
stated, but is a little more difficult.)

$(\Rightarrow)$ Given a tree $T$, we know that $T$ has $n$ vertices and $n-1$
edges, thus $\sum d_i = 2(n-1)$ by the fact that $\sum_{v\in G} \deg(v) = 2m$
on any graph.

Now, since any tree has at least two end vertices $v_n, v_{n-1}$, it follows
that $d_n = d_{n-1} = 1.$

$(\Leftarrow)$ We give an explicit construction of a tree $T$. Suppose that
there are $k$ 1's in the degree sequence. Since the sum is $2n-2$ and all the
$d_i \ge 1$ we have $k \ge 2.$ We draw a path $x, u_1, \dots, u_{n-k}, y.$ For
$1 \le i \le n-k,$ attach $d_i-2$ vertices of degree 1 to $u_i.$ The resulting
graph is a tree $T$ (there may be many trees with the degree sequence, but we
only have to produce one of them).

In $T$, it follows from the construction that $\deg u_i = d_i$ for $i =
1,\dots, n-k$. So we need only check that $T$ has the correct number of leaves,
namely $k$ leaves. Observe that with our notation, we have $d_j = 1$ for $j =
n-k+1, \dots, n.$ Now, $x$ and $y$ are leaves, so the total number of leaves is
\[2 + \sum_{i=1}^{n-k} (d_i-2) = 2-2(n-k)+\sum_{i=1}^n d_i - \sum_{i=n-k+1}^n
d_i = 2-2(n-k)+2(n-1)-k = k\] as required.

\medskip
%%%%%%%%%%%%%%%%%%%%%%%%%%%%%%%%%%%%%%%%%%%%%%%%%%%%%%%%%%%%%%%%%%%%%%%%%%%%%%%%
\1 Show that a tree has at most one perfect matching.

\medskip
\noindent \emph{Solution.} Let $T$ be a tree on $n$ vertices. We give a proof
by induction on $n.$ We verify the cases $n=1$ and $n=2$. A tree on 1 vertex
has no perfect matching and a tree on two vertices has one perfect matching.

Let $T$ be a tree on $n$ vertices and let $v$ be a leaf of $T$. In any perfect
matching, $v$ must be matched to its unique neighbor $u$. The remainder of any
matching is a matching in $T-\{u,v\}$. Thus any perfect matching in $T$ must
contain the edge $uv$ so that the number of perfect matchings in $T$ equals the
number of perfect matchings in $T-\{u,v\}$.

The graph $T-\{u,v\}$ is a forest, i.e.~a disjoint union of trees. By
induction, each component has at most one perfect matching. The number of
perfect matchings in a graph is the product of the number of perfect matchings
in each component, therefore $T-\{u,v\}$ has at most one perfect matching and
therefore so does $T$.

(This is actually a proof by strong induction.)

\medskip
%%%%%%%%%%%%%%%%%%%%%%%%%%%%%%%%%%%%%%%%%%%%%%%%%%%%%%%%%%%%%%%%%%%%%%%%%%%%%%%%
\1 Let $G$ be a bipartite graph with bipartitions $X$ and $Y$ such that $|X|
\le |Y|$. Suppose that $|N(S)| > |S|$ for every nonempty subset $S$ of $X$.
Show that for each edge $e$ of $G$, there exists a matching $M$ such that $e\in
M$ and $|M| = |X|.$

\medskip
\noindent \emph{Solution.} Let $xy$ be an edge of $G$ with $x\in X$ and $y\in
Y$ and put $G' := G-x-y.$ Each set $S \subset X-\{x\}$ loses at most one
neighbor when $y$ is deleted. Since $|N(S)| > |S|$ by assumption, we have
$|N_{G'}(S)| \ge |N_G(S)|-1 \ge |S|.$ Therefore $G'$ satisfies the condition of
Theorem 8.3 (Hall's theorem) and therefore $G'$ has a matching $M'$ such that
$|M'| = |X|-1$. Then $M := M' \cup \{xy\}$ is the required matching which
contains $xy$ and is such that $|M| = |X|$.

\medskip
%%%%%%%%%%%%%%%%%%%%%%%%%%%%%%%%%%%%%%%%%%%%%%%%%%%%%%%%%%%%%%%%%%%%%%%%%%%%%%%%
\1 Two people play a game on the graph $G$, alternately choosing distinct
vertices. Player 1 starts by choosing any vertex $v_1$. Player 2 then chooses
any vertex $w_1$ adjacent to $v_1$. Each subsequent choice must be adjacent to
the preceding choice of the other player. Together, the players follow a path
through the graph. The last player able to move wins.

Show that if $G$ has a perfect matching, then Player 2 has a winning strategy.

\medskip
\noindent \emph{Solution.} Let $M$ be a perfect matching of $G$. Thus $|G|$ is
necessarily even. Whenever Player 1 chooses a vertex $v$, Player 2 chooses the
vertex $w$ which is matched to $v$ by $M$, i.e.~such that $vw\in M$. Since $M$
is a perfect matching the sequence of moves is $v_1 w_1 v_2 w_2 \cdots v_k w_k$
with $|G| = 2k$. Thus Player 2 always has the last move.

\medskip
%%%%%%%%%%%%%%%%%%%%%%%%%%%%%%%%%%%%%%%%%%%%%%%%%%%%%%%%%%%%%%%%%%%%%%%%%%%%%%%%
\1 Show that every graph $G$ with $\delta(G) \ge (n+k-2)/2$ is $k$-connected,
where $k \le n-1$.

\medskip
\noindent \emph{Solution.} If $G$ is not $k$-connected, then we can delete a
set $S$ of $k-1$ vertices to get a disconnected subgraph $H$. Let $v \in H$.
Since $v$ has at most $k-1$ neighbors in $S$, we have that \[\deg_{H}(v) \ge
\delta(G) - k + 1 \ge (n-k)/2.\] Thus each component of $H$ has at least
$1+(n-k)/2$ vertices. Since $H$ is disconnected, it has at least two
components, and therefore $H$ has at least $n-k+2$ vertices. But $n = |G| = |H|
+ |S| \ge (n-k+2) + (k-1) > n$. This is a contradiction.

\medskip
%%%%%%%%%%%%%%%%%%%%%%%%%%%%%%%%%%%%%%%%%%%%%%%%%%%%%%%%%%%%%%%%%%%%%%%%%%%%%%%%
\1 A directed acyclic graph (DAG) is a directed graph $G$ without any directed
cycles. A topological ordering of a directed graph $G$ is an ordering of its
vertices $v_1,v_2,\dots,v_n$ such that if $(v_i,v_j)$ is an edge of $G$ then $i
< j$.

Show that $G$ is a DAG if and only if $G$ has a topological ordering.

\medskip
\noindent \emph{Solution.} $(\Rightarrow)$ If $G$ has no cycles, then some
vertex $v$ has outdegree 0. Let $v_n := v$ be the last in the ordering. Now
$G-v$ has no cycles and we may repeat the previous step to find $v_{n-1}$. And
so on. Suppose that we have completed the execution of this algorithm. Then by
construction, $v_j$ is such that it has no successors among $v_1, \dots,
v_{j-1}$. This shows that $(v_i,v_j) \in E(G) \Rightarrow i < j$.

$(\Leftarrow)$ We are given a topological ordering $v_1,v_2,\dots,v_n$ on $G$.
Suppose that $G$ has cycle $C$. Let $v_i$ be the lowest-indexed vertex on $C$
and let $v_j$ be the vertex on $C$ just before $v_i$. Thus $(v_j,v_i)$ is an
edge. By our choice of $i$ we have $j > i$ which contradicts the fact that
$v_1,v_2,\dots,v_n$ is a topological ordering.

\medskip
%%%%%%%%%%%%%%%%%%%%%%%%%%%%%%%%%%%%%%%%%%%%%%%%%%%%%%%%%%%%%%%%%%%%%%%%%%%%%%%%
\1 Show that there is a tournament on $n$ vertices such that the indegree is
equal to the outdegree for every vertex if and only if $n$ is odd.

\medskip
\noindent \emph{Solution.} $(\Rightarrow)$ We prove the contrapositive. If $n$
is even, then $n-1 = \mathrm{outdeg}(v) + \mathrm{indeg}(v)$ is odd, thus the
summands cannot be equal integers.

$(\Leftarrow)$ If $n$ is odd, then every vertex of $K_n$ has even degree, and
therefore there exists an Eulerian circuit on $K_n$ by Theorem 6.1. Orienting
the edges of $K_n$ in the forward direction by following an Eulerian circuit
produces tournament we are looking for.

\medskip
%%%%%%%%%%%%%%%%%%%%%%%%%%%%%%%%%%%%%%%%%%%%%%%%%%%%%%%%%%%%%%%%%%%%%%%%%%%%%%%%
\1 On a chessboard, a knight can move from a square at coordinate $(a,b)$ to a
square at coordinate $(a',b')$ if and only if $(|a-a'| = 1$ and $|b-b'| = 2)$
or $(|a-a'| = 2$ and $|b-b'| =1)$. This is just the familiar L-shaped knight's
move. Show that no $4\times n$ chessboard has a knight's tour: a traversal by
knight's moves that visits each square once and returns to the start. (Hint: If
a graph $G$ is Hamiltonian then $k(G-S) \le |S|$ for all nonempty $S \subset
G$.)

\medskip
\noindent \emph{Solution.} Let $G$ be the graph with a vertex for each square
and an edge for each pair of squares whose positions differ by a knight's move.
The existence of the knight's tour is equivalent to the existence of a
Hamiltonian cycle in $G$.

There are 4 rows. We observe that every neighbor of a white square in the top
and bottom rows is a black square in the middle two rows. Let $S$ be the set of
black squares in the middle two rows, so that $|S| = n$. In $G-S$, the white
squares in the top and bottom rows are thus $n$ isolated vertices. Now $|G-S| =
3n$, and in $G-S$ we have found $n$ isolated vertices. The remaining $2n$
vertices in $G-S$ must belong to at least one other connected component. Thus
$G-S$ has at least $n+1$ components. Therefore $k(G-S) > |S|$ and $G$ is not
Hamiltonian, as required.

\medskip
%%%%%%%%%%%%%%%%%%%%%%%%%%%%%%%%%%%%%%%%%%%%%%%%%%%%%%%%%%%%%%%%%%%%%%%%%%%%%%%%

\end{outline}

\end{document}


